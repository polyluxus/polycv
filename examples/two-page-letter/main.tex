\documentclass[
% draft,
  xcolor=dvipsnames,
  primary=Blue,
  biblatex,
  signcv,
  14pt,a4paper
]{polycv}

\usepackage{mwe}          % Only for this example
\usepackage{filecontents} % Only for this example
\errorcontextlines=20

% Fonts declaration 
\usepackage[T1]{fontenc}
\usepackage[utf8]{inputenc}
\usepackage{libertine}

% bibliography
\usepackage[backend=biber,%
            style=numeric-comp,
            style=chem-angew,
            subentry=true,
            sorting=ydnt,% sorted by year, descending
]{biblatex}
\begin{filecontents}{\jobname.bib}
@BOOK{example-book,
  author    = {Example Author},
  title     = {Example Book Title},
  publisher = {Example Publisher},
  year      = {2019},
}

@ARTICLE{example-article,
  author = {Example Author},
  title = {Example Title},
  journal = {J. Examples},
  year = {2019},
  volume = {0},
  pages = {1227},
}

@Thesis{example-thesis,
  author    = {Example Author},
  title     = {Example Thesis Title},
  school    = {University of Examples},
  year      = {2019},
  type      = {PhD-Thesis},
  address   = {Example Town},
}

@CONFERENCE{example-talk,
  author    = {Example Author},
  title     = {Example Thesis Title},
  booktitle = {Book of Abstracts, 1st Annual Example Meeting},
  year = {2019},
}

\end{filecontents}
\addbibresource{\jobname.bib}

\usepackage[english]{babel}
\usepackage{multicol}

% Start customisation for CV

% Customize lengths for polycv:
% \setpolycvheaderheight{5cm}  % Default: 4.0 cm
% \setpolycvfooterheight{2cm}  % Default: 0.5 cm
% \setpolycvmargins{0.8cm}     % Default: 1.0 cm
% \setpolycvhintcolumnwidth{3cm} % Default: 2.7 cm
% \setpolycvhintcolumnsep{1cm} % Default: 0.2 cm
% \setpolycviconspace{0.6cm}     % Default: 0.8 cm
% \setpolycventryraggedright   % Default: justified

% Fill some content
\title{Curriculum Vitae}
\author{Example Author}
\position{Title / Position}
\street{Street 123}
\location{Town, Country}
\email{me@mail.com}
\phone{+1 234 56 78 900}
\mobile{+1 234 56 78 999}
\orcid{0000-0001-0002-0003}
\github{ex-ample}
\signatureimage{example-image-16x9}

\begin{document}
%Open the front page environment
\begin{polycvfirstpage}
% Cite everything first. This works because the bibliography settings (above) sort chronologic.
\nocite{*}%
% Define what goes into the sidebar (This should be done first.)
\begin{polycvsidebar}
\small

\includegraphics[width=1.0\linewidth]{example-image-1x1}\\[2ex]

\section{Personal Details}
% Universally usable line:
\polycvitemline{\faStar}{01. 01. 1984,\newline Birthplace, Country}
\polycvlineaddress
\polycvlineemail
\polycvlinephone
\polycvlinemobile
\polycvlineorcid
\polycvlinegithub
\null

\section{Languages}
\polycvlanguage{English (native)}{1}
\polycvlanguage{Leetspeak (B2)}{0.7}

\section{Skills}
\polycvskill{\LaTeX}{0.8}
\polycvskill{Reading}{0.9}
\polycvskill{Writing}{0.7}
\polycvskill{Calculating}{0.3}
\polycvskill{Gaming}{0.95}

\end{polycvsidebar}%
% First page content goes here
\section{Experience}
%\cventry{years}{degree/jobtitle}{institution/employer}{localization}{optional: grade/...}{optional: comment/job description}
\polycventry{2018 - 2019}
{Writer}
{Example Publisher}
{Example Place}
{}
{First reading things, and then writing about those things.}

\polycventry{2017 - 2018}
{Reader}
{Example Company}
{Example Place}
{}
{Reading things.}

\section{Education}
\polycventry{2010 - 2017}
{Student}
{Example School}
{Example Place}
{Grade: excellent}
{Learning how to read and write.}

\section{Languages}
\polycvlanguage{English (native)}{1}
\polycvlanguage{Leetspeak (B2)}{0.7}

\section{Skills}
\begin{multicols}{2}
\polycvskill{\LaTeX}{0.8}
\polycvskill{Reading}{0.9}
\polycvskill{Writing}{0.7}
\polycvskill{Calculating}{0.3}
\polycvskill{Gaming}{0.95}
\end{multicols}

\section{More Skills}
% generic line \polycvline{<hintcol>}{<description>}
% levels with boxes \polycvlevelssquared[<width>]{<num>} or \polycvlevelscircled[<width>]{<num>}
\polycvline{\polycvlevelssquared[0.7\linewidth]{4}}{Reading emails}
\polycvline{\polycvlevelscircled[0.7\linewidth]{3}}{Writing emails}
\polycvline{\polycvlevelssquared[0.7\linewidth]{3}}{Reading emails}
\polycvline{\polycvlevelscircled[0.7\linewidth]{2}}{Writing emails}
% close the first page
\end{polycvfirstpage}
% open the second page (changes margin, repeatable)
\begin{polycvpage}
\defbibheading{bibliography}[Publications and Talks]{\section{#1}}
\printbibliography
% close the page
\end{polycvpage}
% open the environment for the cover letter (this can be done first, too)
\begin{polycvletter}[\polycvaddress\\\polycvmobile\\\polycvemail]{Subject}{Company\newline Address}
  \opening{Dear Example,}

  \blindtext[1-3]

  \closing{Sincerely,}
  \ps{\blindtext[1]}
\end{polycvletter}
\end{document}

